\section{Sintaxis}

	\begin{itemize}
	\item Después de \textit{if / for / def / ...} $\rightarrow$ :
	\item Final de línea = final de expresión. Para dividir una sentencia en varias líneas: $\backslash$. Las siguientes líneas se pueden indentar de $\forall$ manera. (Las expresiones entre corchetes/llaves/paréntesis también pueden ocupar varias líneas sin  $\backslash$)  
	\item Final de indentación \footnote{Cualquier indentación es válida mientras sea coherente en todo el bloque} = final de bloque 
	\end{itemize}

	\subsection{Condiciones: \textit{if - elif - else}}
	\noindent Sintaxis:
	\begin{tabbing}
    ---- \= ---- \= \kill
	\> \textit{if} condición:\\
	\>\> expresión1 \\
	\>\> expresión2 \\ \\
	\>\textit{elif} condición: \\
	\>\> expresión alternativa1 \\
	\> \textit{else}: \\
	\>\> expresión alternativa2 
	\end{tabbing}

	\begin{itemize}
	\item Si sólo tiene una condición puede hacerse en una única línea: \textit{if} condición: expresión 	
	\item No hay \textit{case}, se usan diccionarios
	\item Sintaxis alternativa: \textit{A = Y if X else Z}\footnote{Y ? X : Z de C y Java} equivale a:
	\begin{tabbing}
    ---- \= ---- \= \kill
	\> \textit{if} X:\\
	\>\> A = Y\\
	\> \textit{else}: \\
	\>\> A = Z
	\end{tabbing}

	\end{itemize} 

	\subsection{Bucles}
		
	\noindent Para parar un bucle $\rightarrow$ Ctrl + C

		\subsubsection{Bucles \textit{for}}
		\noindent Sintaxis:
		\begin{tabbing}
		---- \= ---- \= \kill		
		\> \textit{for} contador \textit{in} objeto : \\
		\>\> expresiones \\
		\> \textit{else}: \\
		\>\> expresiones
		\end{tabbing}

		\begin{itemize}
		\item Se puede iterar en strings, listas, tuples ...
		\item Se le pueden poner \textit{break}s y \textit{continue}s (ver \ref{while})
		\item Se puede vectorizar en lugar de usar bucles \textit{for} para ganar velocidad. Ejemplo:
			\begin{enumerate}
			\item \textit{cuadrados = [x**2 for x in [1, 2, 3, 4]]}(puede anidarse un \textit{if})
			\item \textit{cuadrados = [] \# lista vacía}     
			\begin{tabbing} % Indentación
         	---- \= ---- \= \kill  % Definición
			\textit{for} x \textit{in} [1,2,3,4]:\\
			\>\textit{cuadrados.append(x**2)}  % \> para que indente; \>\> indenta al segundo nivel
			\end{tabbing}
			\end{enumerate}	
		\end{itemize}

		\subsubsection{Bucles \textit{while}}\label{while}
		\noindent Sintaxis:
		\begin{tabbing}
	    ---- \= ---- \= \kill
		\> \textit{while} condición:\\
		\>\> expresión1 \\
		\>\> expresión2 \\ \\
		\>\textit{try}: (como un \textit{try - catch} de Java)\\
		\>\textit{except}: \\
		\>\> maneja error 
		\end{tabbing}

		\noindent Excepciones dentro del while:

		\begin{tabbing}
	    ---- \= ---- \= \kill
		\> \textit{while} condición1:\\
		\>\> expresiones \\
		\>\textit{if} condición2 : \textit{break \# sale del while, no ejecuta else}\\
		\>\textit{if} condición3 : \textit{continue \# vuelve al while}\\
		\> else: \\
		\>\> expresiones \# \textit{ejecuta si no ha ejecuta un break} 
		\end{tabbing}

		\subsubsection{Bucles \textit{do while}}
		\begin{itemize}
		\item No hay bucles \textit{do while}
		\item Para hacer un equivalente a un \textit{do while}
		\begin{tabbing}
	    ---- \= ---- \= \kill
		\> \textit{while} True:\\
		\>\> expresiones \\
		\>\> \textit{if} exitTest() : break \\
		\end{tabbing}
		\end{itemize}

	\subsection{Operadores}

	\begin{itemize}
	\item \textbf{Suma} $\rightarrow$ + , funciona como suma para el caso de los números y como concatenación para los strings y listas (L = [1,2] $\rightarrow$ L = L +[3] $\rightarrow$ L = [1,2,3])

	\item \textbf{Multiplicación} $\rightarrow$ * , multiplicación entre números y repetición para listas y strings. Ej: 'Spam'*3 $\rightarrow$ 'SpamSpamSpam'

	\item \textbf{División} $\rightarrow$ /, división entera entre int (truncando \footnote{Para que divida normal entre dos enteros .0 o hacer un cast}) y división entre cualquier combinación float - int.

	\item \textbf{Módulo} $\rightarrow$ \%, el resto en una división entera

	\item \textbf{Exponente} $\rightarrow$ **

	\item \textbf{Slice} $\rightarrow$ objeto[\textit{desde donde:paso: hasta donde}]. No coge el último elemento. Si el último número es negativo va hacia atrás. \footnote{A[:] coge todos los elementos}

	\item \textbf{Cast}: convertir un tipo de objetos en otro. En entero: \textit{int(dato)}. En real: \textit{float(dato)}. En string: str(dato). En lista: list(tuple). En tuple: tuple(lista). 

	\item Operador \textbf{in}: \textit{if} x \textit{in} y. Devuelve True si x pertenece a y. Ejemplos: 
		\begin{itemize}
		\item\textit{'m' in 'Spam'} $\rightarrow$ True
		\item Para leer un fichero: \textit{for} line \textit{in} \textit{open('file.txt')}:
		\item Para iterar en un diccionario: \textit{for} key \textit{in} \textit{D.keys()}: o \textit{for} key \textit{in} D :
		\end{itemize}
	\end{itemize}
