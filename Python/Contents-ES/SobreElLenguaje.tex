\section{Sobre el lenguaje} % INDENTAR EL 1.1.1

	\begin{itemize}
	\item Interpretado
	\item Indentación obligatoria
	\item Distingue mayúsculas - minúsculas
	\item No hay declaración de variables (\textit{dynamic typing})
	\item Orientado a objetos  
	\item Garbage colector: quita los objetos a los que no haga referencia nada
	\item Comentarios $\rightarrow$ \#  %para meter fórmula matemática entre $-$ 
	\item Imprimir en pantalla $\rightarrow$ print a,b (si no se quiere salto de línea coma al final). Para darle formato (igual que C): 
		\begin{itemize}
		\item Float: \%dígitos.dígitos f variable
		\item String: \% s variable
		\item Notación científica: \%dígitos.dígitos E variable
		\item print \textit{'\%s letra \%s' (a,b)} escribe \textit{a letra b}. Identifica los \% con los elementos del tuple.
		\end{itemize}
	\item Nombres de variables:
		\begin{itemize}
		\item Solo pueden empezar por \_ o letra, luego cualquier carácter
		\item Diferencia mayúsculas y minúsculas (\textit{case sensitive})
		\item No puede ser igual que palabra reservada: \textit{and, del, fo, is, raise, asser, elif, from, lambda, return, break, else, global, not, try, class, except, if, or while, continue, exec, import, pass, yield, def, finally, in, print}
		\end{itemize}
	\item En línea de comandos, la variable \_ guarda el valor de la última operación
	\end{itemize}