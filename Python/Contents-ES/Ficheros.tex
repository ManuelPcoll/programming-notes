\section{Ficheros}

	\begin{itemize}
	\item Para escribir: \\
	\textit{f = open('nombre.txt','w') \# modo escribir \\
	f.write('texto') \\
	f.close() \# se crea el fichero}

	\item Para leer: (se lee como string)\\ 
	\textit{	f = open('nombre.txt')\\
	string = f.read()}\\

	\noindent Dos opciones:
	\begin{tabbing}
	---- \= ---- \= \kill  
	\> \textit{string \# muestra el contenido}\\
	\> \textit{print string \# muestra el contenido con formato}
	\end{tabbing}

	\item Para convertir cosas de un fichero en objetos Python, ejemplos:

		\begin{enumerate}
		\item \textbf{Strings}:\\
		\textit{F = open('file.txt')\\
		line = F.readline() \# lee línea \\ 
		line.rstrip() \# quita \textbackslash n}

		\item \textbf{Números}: '44,44,45 \textbackslash n' \\
		\textit{F = open('file.txt')\\
		line = F.readline() \# lee línea \\ 
		parts = line.split\footnote{string.split('carácter')$\rightarrow$ convierte string en lista de palabras cortando por el carácter indicado. Para convertir lista en string: delimiter.join(lista) (delimiter es el carácter que une)}(',') \# parte por la coma, quedaría ['43','44','45\textbackslash n']\\
		numbers = [int(P)] for P in parts \# convierte a número}

		\item \textbf{Código}: '[1,2,3]\$\{'a':1, 'b':2 \}\textbackslash n'
		\textit{F = open('file.txt')\\
		line = F.readline() \# lee línea \\ 
		parts = line.split('\$') \# parte por \$ \\
		eval(parts[0]) \# convierte a cualquier tipo de objeto (trata como código)
		\# objects = [eval(P) for P in parts] para coger todo. Queda [[1,2,3],{'a':1,'b':2}]}

		\item \textbf{Pickle}:
		Para el caso que \textit{eval()} no sea fiable, se puede usar el módulo \textit{pickle}:\\
		\textit{F = open('file.txt','w')}\\
		import picle \\
		pickle.dump(D,F) \# coge cualquier objeto de la fila\\
		F.close() \\
		\end{enumerate}

	\item Hay que poner \textbackslash n para saltar línea
	\item Si no se cambia el path guarda los ficheros en C:\textbackslash Temp

	\end{itemize}