\section{Introducción}

En este apartado se introduce el funcionamiento general de Erlang, su tipado, sintaxis y filosofía general.

\subsection{Funcionamiento}

Erlang dispone de una máquina virtual donde se ejecutan los programas. Es parecido a lo que hace java con los
bytecodes. Los programas hechos en Erlang tienen que ser compilados para esta máquina virtual.

Los archivos de código fuente tienen la extensión '.erl', de Erlang,  y los compilados '.beam', que proviene
de \textit{Bogdan/Björn's Erlang Abstract Machine}, el nombre de la máquina virtual. Han existido otras
máquinas virtuales pero están en desuso.

Más adelante se profundizará en la compilación.

\subsection{La Shell}

Erlang dispone de una shell. Sirve como emulador para poder ejecutar comandos pero también permite cosas como
editar código en caliente etc. Para entrar: usar el comando \textit{erl}.

\begin{lstlisting}
ekaitz@DaComputa:~$ erl
Erlang R16B03 (erts-5.10.4) [source] [64-bit] [smp:8:8] [async-threads:10] [kernel-poll:false]

Eshell V5.10.4  (abort with ^G)
1>
\end{lstlisting}

La shell está basada en Emacs, usa comandos similares [Ctrl+A] para inicio de línea, [Ctrl+E] para final de
línea, etc.


Como se aprecia arriba, [Ctrl+G] sirve para abortar. Una vez pulsado permite ejecutar unos comandos simples:


\begin{lstlisting}
1>
User switch command
 --> h
  c [nn]            - connect to job
  i [nn]            - interrupt job
  k [nn]            - kill job
  j                 - list all jobs
  s [shell]         - start local shell
  r [node [shell]]  - start remote shell
  q                 - quit erlang
  ? | h             - this message
 -->
\end{lstlisting}

Para terminar las líneas introducir '.', como cuando se pone ';' en MySQL. Las expresiones pueden separarse
por comas pero sólo se mostrará el resultado de la última.

\begin{lstlisting}
1> 2/3,3*4.
12
\end{lstlisting}
