% Copyright (C)  2015  Ekaitz Zárraga Río <ekaitzzarraga@gmail.com>.
%     Permission is granted to copy, distribute and/or modify this document
%     under the terms of the GNU Free Documentation License, Version 1.3
%     or any later version published by the Free Software Foundation;
%     with no Invariant Sections, no Front-Cover Texts, and no Back-Cover Texts.
%     You should have received a copy of the GNU Free Documentation License
%     along with this material.
%     If not, see <http://www.gnu.org/licenses/>.

\section{Módulos}

Un módulo (\textit{module} en inglés) es un conjunto de funciones agrupadas en un único archivo, definido con
un nombre con la extensión \textit{.erl}.

Para ejecutar funciones de un módulo concreto la sintaxis es la siguiente:

\begin{verbatim}
Modulo:Función(Argumentos).
\end{verbatim}

En el siguiente ejemplo \textit{io} sería el módulo, la función \textit{format} y los argumentos el string:
\textit{``Hola Mundo!''}:
\begin{lstlisting}
77> io:format("Hola Mundo!~n").
Hola Mundo!
ok
\end{lstlisting}

En los módulos hay dos tipos de cosas: Funciones y Atributos.

\paragraph{Los atributos} son el conjunto de metadatos que describen el módulo y se definen así:
\begin{verbatim}
-Nombre(Valor).
\end{verbatim}

\begin{itemize}
  \item El atributo mínimo para un módulo es el nombre:
    \begin{verbatim}
    -module(NombreDelModulo).
    \end{verbatim}

  \item Para definir qué funciones serán mostradas al exterior del módulo se utiliza el atributo
    \textit{-export()} \footnote{Aridad: En inglés \textit{arity}. En el análisis
    matemático, la aridad de un operador matemático o de una función es el número de argumentos necesarios
    para que dicho operador o función se pueda calcular.}:
    \begin{verbatim}
    -export([Función1/Aridad, Función2/Aridad...]).
    \end{verbatim}

  \item Para importar funciones de otros módulos:
    \begin{verbatim}
    -import(Modulo,[Función1/Aridad, Función2/Aridad...]).
    \end{verbatim}

  \item Para definir macros: \textit{-define(Nombre,Valor).} que se utilizan \textit{?Nombre}. Por ejemplo,
  la macro \textit{-define(sub(X,Y), X-Y).} utilizada como \textit{?sub(13,5)} tomará el valor \textit{13-5}.
\end{itemize}


\paragraph{Las funciones} se definen de la siguiente manera:
\begin{verbatim}
Nombre(Argumentos)-> Cuerpo.
\end{verbatim}

\begin{itemize}
  \item Nombre: Es un \textit{atom}, así que debe comenzar en minúsculas.
  \item Cuerpo: Expresiones separadas por comas, la última expresión ejecutada será el valor devuelto.
  \item Argumentos: Separados por comas.
\end{itemize}

Todos los módulos tienen (automáticamente) una función \textit{module_info/0} que muestra los metadatos del
módulo. Para llamarla:
\begin{lstlisting}
modulo:module_info().
\end{lstlisting}


Ejemplo de módulo:
\begin{lstlisting}
-module(functions).
-export([head/1,second/1]).

head([H|_]) -> H.
second([_,X|_]) -> X.
\end{lstlisting}

\subsection{Compilación}

Tal y como se ha dicho antes el código fuente de Erlang debe ser compilado para la máquina virtual. Existen
muchas formas de hacerlo:
\begin{itemize}
  \item Desde la shell de Erlang:
    \begin{lstlisting}[gobble=4]
    1> c(modulo).
    {ok,functions}
    \end{lstlisting}
    Si el módulo no se encuentra en la carpeta actual navegar usando \textit{cd(``path'').} y \textit{ls().} o
    introducir el path hasta el archivo.

  \item Desde la shell de unix:
    \begin{lstlisting}[language=bash,gobble=4]
    ekaitz@DaComputa:~$ erlc flags modulo.erl
    \end{lstlisting}

  \item Desde la shell de Erlang o desde un módulo:
    \begin{lstlisting}[gobble=4]
    compile:file(modulo).
    \end{lstlisting}
\end{itemize}




